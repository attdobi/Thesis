%Abstract Page 

\hbox{\ }

\renewcommand{\baselinestretch}{1}
\small \normalsize

\begin{center}
\large{{ABSTRACT}} 

\vspace{3em} 

\end{center}
\hspace{-.15in}
\begin{tabular}{ll}
Title of dissertation:    & {\large  MEASUREMENT OF THE ELECTRON RECOIL}\\
&				      {\large  BAND OF THE LUX DARK MATTER DETECTOR} \\
&				      {\large  WITH A TRITIUM CALIBRATION SOURCE} \\

\ \\
&                          {\large  Attila Dobi, Doctor of Philosophy, 2014} \\
\ \\
Dissertation directed by: & {\large  Professor Carter Hall} \\
&  				{\large	 Department of Physics } \\
\end{tabular}

\vspace{3em}

\renewcommand{\baselinestretch}{2}
\large \normalsize

The Large Underground Xenon (LUX) experiment has recently placed the most stringent limit for the spin-independent WIMP-nucleon scattering cross-section. The WIMP search limit was aided by an internal tritium source resulting in an unprecedented calibration and understanding of the electronic recoil background. Here we discuss corrections to the signals in LUX, the energy scale calibration and present the methodology for extracting fundamental properties of electron recoils in liquid xenon. The tritium calibration is used to measure the ionization and scintillation yield of xenon down to 1 keV, the measurement of scintillation yield is compared to other Compton scatter measurements.  Recombination probability and its event-to-event fluctuation is measured from 1 to 1000 keV, using betas from tritium and Compton scatters from an external $\rm ^{137}Cs$ source. New models are presented for recombination in liquid xenon in an attempt to explain the discrepancies between the current understanding and the observed fluctuations. Finally, the results for the electron recoil band based on the tritium calibration is presented and used to characterize background rejection for the LUX WIMP search.
