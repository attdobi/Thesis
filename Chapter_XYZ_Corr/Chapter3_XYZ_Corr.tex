%Chapter 3

\renewcommand{\thechapter}{3}

\chapter{XYZ Correction}

\section{$\rm^{83m}Kr$ Calibration Source}

Throughout the science run periodic $\rm^{83m}Kr$ injections were performed to calculate and monitor position dependent corrections. $\rm ^{83m}Kr$ is produced from the decay Rb, the parent Rb is housed in charcoal. The daughter $\rm{83m}Kr$ is continually produced and has a half-life 1.8 hours, it is introduced when needed into the LUX detector by flushing the charcoal housing with xenon along the main circulation path. The source is described in more detail in ref[]. The relatively short half-life allows for several injections per week without interrupting WIMP search data taking. Once injected the source is uniformly mixed into the liquid xenon within a matter of minutes and can be used to calculate corrections for the XYZ the response of the detector. Thus, the mono-energetic gamma emitted from 83mKr is a powerful tool for tracking detector parameters and detector response calibrations over the course of the science run.
Figure 1 shows the distribution of 83mKr events in the LUX detector thirty minutes after the injection.


...These corrections include the the free electrons lifetime and photon detection efficiency vs. XYZ. 


$\rm^{83m}Kr$ first emits a 32.1 [keV] gamma followed by a 9.4 [keV] with a half life of 154 [ns] between the two (refs). The combined signal (41.6 [keV]) is found by the pulse finder in the majority of cases, using the standard WIMP search pulse gap setting of 500 ns. 