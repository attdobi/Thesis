
%Chapter 1

\renewcommand{\thechapter}{1}

\chapter{Introduction}

\section{Dark Matter}

	WIMPs ... ... ...
%\cite{Agrawal1,Bloembergen,Shen,Butcher,Boyd} -
%1.1

\section{Dark Matter Detectors}

Overview of field

\section{Outline of Thesis}

In Chap.\ 2, we present the results of a computational study of the
influence of stochasticity on the dynamical evolution of multiple 
four-wave-mixing processes in a single mode optical fiber with spatially
and temporally $\delta$-correlated phase noise. A generalized nonlinear
Schr\"odinger equation (NLSE) with stochastic phase fluctuations along the
length of the fiber is solved using the Split-step Fourier method
(SSFM). Good agreement is obtained with previous experimental and
computational results based on a truncated-ODE (Ordinary Differential
Equation) model in which stochasticity was seen to play a key role in
determining the nature of the dynamics. The full NLSE allows for
simulations with high frequency resolution (60\,MHz) and frequency span (16
THz) compared to the truncated ODE model (300\,GHz and 2.8\,THz,
respectively), thus enabling a more detailed comparison with
observations. A physical basis for this hitherto phenomenological phase
noise is discussed and quantified.

In Chap.\ 3, we discuss the implications of spontaneous and stimulated
Raman scattering on the project discussed in Chap.\ 2, namely, the dynamical evolution of 
stochastic four-wave-mixing processes in an optical fiber.
The following question is asked - can stimulated Raman scattering be a mechanism by which
adequate multiplicative stochastic phase fluctuations are introduced in the 
electric field of light undergoing four-wave-mixing as? Adequately checked numerical
algorithms of stimulated Raman scattering (SRS), spontaneous Raman generation and intrapulse 
Raman scattering (IRS) are used while exploring this issue. The algorithms are described in detail, as also are 
the results of the simulations. It is found that a 50-meter length of fiber (as used in the experiments),
is too short to see the influence of Raman scattering, which is found to eventually 
dominate for longer fiber lengths.

In Chap.\ 4, self- and cross-phase modulation (XPM) of femtosecond pulses ($\sim$ 810
nm) propagating through a birefringent single-mode optical fiber ($\sim$ 6.9
cm) is studied both experimentally (using GRENOUILLE - Grating Eliminated 
No Nonsense Observation of Ultrafast Laser Light Electric Fields) 
%(using second harmonic
%generation-frequency resolved optical gating or SHG-FROG) 
and numerically
(by solving a set of coupled nonlinear Schr\"odinger equations or
CNLSEs). An optical spectrogram representation is derived from the
electric field of the pulses and is linearly juxtaposed with the
corresponding optical spectrum and optical time-trace. The effects of
intrapulse Raman scattering (IRS) are discussed and the question whether 
it can be a cause of asymmetric tranfer of pulse energies towards longer 
wavelengths is explored. The simulations are shown to be in good qualitative 
agreement with the experiments. Measured input pulse asymmetry, when incorporated 
into the simulations, is found to be the dominant cause of output spectral 
asymmetry. \renewcommand{\baselinestretch}{1} \small\footnotesize
\footnote{These averages are reported
for $45$ `detailed occupational codes', which is an intermediate
occupational classification (between two and three-digit codes)
given by the Current Population Survey (CPS).}
\renewcommand{\baselinestretch}{2} \small\normalsize
The results indicate that it is possible to modulate short pulses both temporally and spectrally by passage through polarization maintaining 
optical fibers with specified orientation and length. The modulation technique is very direct and straightforward. No frequency components of the broadband pulse have to be rejected as the entire spectrum is uniformly modulated. The technique is flexible as the modulation spacing can be varied by varying the fiber length.

Chapter 5 provides the conclusion to the thesis.



%% Useful LATEX tips below:
`
\begin{comment}

\section{Theorems}

\newtheorem{theorem}{Theorem}[chapter]
\begin{theorem}
This is my first theorem.
\end{theorem}


\section{Axioms}
\newtheorem{axiom}{Axiom}[chapter]
\begin{axiom}
This is my first axiom.
\end{axiom}




\begin{axiom}

This is my second axiom in chapter 1.
\end{axiom}

\section{Tables}

This is my table. 

\renewcommand{\baselinestretch}{1}
\small\normalsize

\begin{table}[h]
\caption[Short title]{Overview of test cases used in this study.}
\begin{center}
\begin{tabular}{|c|c|c|c|}
\hline
Test & Quality & Setpoint & Manipulated \\
case & variable (QV) & for QV & variables (MVs)\\
\hline \hline
TE & G/H ratio & 1.226 & D-feed SP and Reactor Level SP\\
AZ & xB($H_2O$) & & Reflux flow and $5^{th}$ Tray temperature SP\\  
\hline
\end{tabular}
\end{center}
\label{test_over}
\end{table}

\renewcommand{\baselinestretch}{2}
\small\normalsize

My table is shown above.   Normally it is double-spaced but I have inserted a command (marked in blue) to make it single-spaced and then inserted a command (again in blue) to change the text back to double-spacing.


\

\subsection{Adding Extra Space between Text and Horizontal Lines}

\renewcommand{\baselinestretch}{1}
\small\normalsize



\begin{table}[h]
\caption{Table with Extra Space between the Text and Horizontal Lines.}
\begin{center}
\begin{tabular}{|p{.5in}|p{1in}|c|p{2.25in}|}
\hline
Test case& Quality variable QV)& Setpoint for QV & Manipulated  variables (MVs)\\
\hline \hline
TE & G/H ratio & 1.226 & D-feed SP and Reactor Level SP\\ \hline
AZ & xB($H_2O$) & & Reflux flow and $5^{th}$ Tray temperature SP \\
\hline
\end{tabular}
\end{center}
\label{test_over}
\end{table}

\renewcommand{\baselinestretch}{2}
\small\normalsize

The line \begin{verbatim}\usepackage{tabls}\end{verbatim} must be inserted in the preamble of your document.
The table is set up to be single-spaced by \begin{verbatim} \renewcommand{\baselinestretch}{1} \small\normalsize\end{verbatim} before \begin{verbatim}\begin{table}\end{verbatim}.  I set the first, second, and fourth columns as paragraphs, .5in, 1in, and 2.25in wide, respectively.  I then adjusted the separation between the words and the horizontal lines to 5ex by also adding \begin{verbatim}\setlength{\tablinesep}{5ex}\end{verbatim} before the \begin{verbatim}\begin{table}\end{verbatim} command.

After typing the table I change the document to be double-spaced from this point on.


\newpage


\subsection{Numbering Figures}

If you wish your figures to be numbered 1-100 without any reference to the chapter (e.g., Figure 1.1, 2.1, etc.), change the first line of your mainthesis.tex file to read \begin{verbatim}"\documentclass[12pt]{thesis-2}".\end{verbatim}  

\subsubsection{This is a Subsubsection}

This is my first subsubsection in Chapter 1.


\section[Short Titles]{Short Titles in the Table of Contents, List of Figures, or List of Tables}

The Table of Contents, List of Figures, or List of Tables usually show the entire title of a section, subsection, etc. or table, or the entire caption of a figure.  If you put a short title in square brackets after \begin{verbatim} \section, \table, or \figure, \end{verbatim} the short title will show in your Table of Contents or lists.

\renewcommand{\baselinestretch}{1}
\small\normalsize

\begin{verbatim}
\section[Short Title]{Title of Section} 
\subsection[Short Title]{Title of Subsection} 
\end{verbatim}

or when using a caption in a figure or table
\begin{verbatim}
\caption[Short Caption]{Full text of the caption.}
\end{verbatim}

\renewcommand{\baselinestretch}{2}
\small\normalsize



\section{LaTeX -- A Typesetting Program}

A 13-page explanation of some of the features of LaTeX can be downloaded from http://www.jgsee.kmutt.ac.th/exell/General/LaTeX.html.


\section{Using Bibtex}

Using Bibtex with Latex documents is not difficult.  The bulk of the work is organizing your bibtex file, which is a data base compiled by you of the articles, books, etc. which you use in the bibliographies or reference sections of your publications.  

I have linked several files to this webpage, which will be helpful when you are using Bibtex.  These files can be downloaded from http://www.ireap.umd.edu/ireap/theses/bibtex.  Please read the file "BibtexInstructions.pdf".  The first two pages explain how to set up and run Bibtex; the remaining pages were taken from a published article and show how the references were cited in the .tex file.   The files BibtexInstructions.tex, Galactic.bib, Dottie.bib are the original .tex files used for BibtexInstructions.pdf.  The file BibtexSamples.tex contains examples of the information needed for the various publications you wish to reference (e.g., articles in refereed journals, books, unpublished articles, conference proceedings, etc.).

If you have questions concerning Bibtex, please contact me at 301-405-4955 or dbrosius at umd.edu.


\section{APS Physical Review Style and Notation Guide}

The following style guide may be downloaded from The American Physical Society at http://forms.aps.org/author/styleguide.pdf:  Physical Review Style and Notation Guide, published by The American Physical Society, compiled and edited by Anne Waldron, Peggy Judd, and Valerie Miller, February 1993.  It may be old, but it is very useful.
 
 \end{comment}
