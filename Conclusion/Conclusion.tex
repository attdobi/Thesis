\renewcommand{\thechapter}{8}

\chapter{LUX and Beyond}
\label{Ch:End}

\section{LUX 2013 Science Run Reanalysis}

The LUX collaboration is currently working on an updated analysis of the 2013 science run, initially published in \cite{LUX_PRL}. The reanalysis will include an improved measurement of the ER band using tritium data along with an improved calibration of the NR mean using a DD neutron generator. The new measurements of background discrimination which will be used is described in section \ref{sec:ER_Discrim}. The tritium calibration set contains twenty times the statistics of the calibration used for the initial LUX WIMP result \cite{LUX_PRL}. The high statistics tritium calibration allows for the ER discrimination pdf used in the profile likelihood analyst to be purely data-driven PDF. Past experiments have assumed Gaussian behavior about the ER mean, however the tritium calibration has unveiled  non-Gaussian tails of the ER band past three sigma of the mean (for energies between 1 to 10 $\rm keV_{ee}$).

In addition to ER discrimination, described in chapter \ref{Ch:7}, the updated WIMP analysis will benefit from improved modeling using the data presented in chapters \ref{Ch:E_Scale_Cal}, \ref{Ch:Flucs}, \ref{Ch:LYQY}. The scintillation and ionization yields measured from the tritium calibration is currently being used to improve upon the NEST package. With the improvements the low energy depositions of $\rm \gamma$ and $\rm \beta$ backgrounds can be precisely converted into of the observables (S1 and S2). 


\section{Fundamental Properties of Liquid Xenon}
In chapter \ref{Ch:Flucs} we presented a method by which event-to-event fluctuations caused by the liquid xenon can be separated from the fluctuations inherent to the detector. It was found that the fluctuations observed can not be explained in terms of just ion-electron pair recombination statistics, currently used in \cite{NEST_2013}. In liquid xenon the variance in the number of recombined ions appears to grow like the number of ions squared, discrediting modeling based on a binomial recombination process. Future experiments with lower photon detection thresholds will be able to probe the puzzle of recombination using the tritium calibration source. By observing ever lower energy deposits as the number of ion-electron pairs produces drops to one it may be found that recombination fluctuation do indeed tend to that of a binomial process.

In chapter \ref{Ch:LYQY} the scintillation and ionization yield of xenon was constrained and found consistent with Compton scatter measurement, both having rather large errors. The errors on the yields measured in the LUX detector using tritium were dominated by the constant on gains $\rm g_1$ $\rm g_2$. In future calibration we plan to improve the constraints on the gains, and hope to measure the light yield and charge yield to within 5\% all the way down to the energy threshold. This represent a significant improvement on the error bars that Compton scattering measurements could hope to achive. 

\section{Internal Calibration Sources}
The tritiated methane source developed by LUX solves the problem of ER calibration for any detector size in the foreseeable future. No matter how big a detector gets, the self-shielding is overcome by mixing the source with the xenon and then removing it. The implementation of the tritiated-methane source is also an important proof of principle that even a long lived radio active isotope can be injected into a xenon detector for calibration purposes. As a complement to the tritiated-methane calibration source, we have developed a method track trace impurities in the xenon gas to the parts-per-trillion ($\rm 10^{-12}$ level, with a specific focus on methane and krypton detection \cite{coldtrap}, \cite{Dobi_CH4}, \cite{Kr_ppt_Dobi}. Having such a system is what ultimately allowed for the use of the source with the LUX detector. We could characterize both the removal and diffusion of natural-methane for the LUX detector its-self, with no reliance of diffusion modeling into plastic components. 

Thus, for any radioactive isotope the implementation used for the tritium source can be transferred over. Any radioactive isotope can be introduced into the xenon as long as its removal, either chemically or through distillation, can be tracked by the gas analysis system. We have specifically focused on methane for the tritated-methane source, which is also chemically identical to methane with a carbon-14, another potential calibration source. There may be a variety of molecules that could be radio-tagged . As long as the molecule can be observed by the gas analysis system the purification efficientcy and diffusion rate can be measured and thus carefully used as a calibration source for a xenon detector. 

