\documentclass[preprint,12pt]{elsarticle}
\usepackage{amsmath}
\usepackage{latexsym}
\usepackage{graphicx}
\usepackage{xspace}
\usepackage{color}
\usepackage{array}
\usepackage{multirow}
\usepackage{hyperref}
\usepackage{wasysym}
\usepackage{verbatim}
\usepackage[font=normalsize]{caption}
\usepackage{mathtools}
\usepackage{relsize}
\addtolength{\oddsidemargin}{-.875in}
\addtolength{\evensidemargin}{-.875in}
\addtolength{\textwidth}{1.75in}
%\addtolength{\voffset}{-70pt}
%\addtolength{\textheight} {120pt}

\newcommand{\Otwo}{\ensuremath{\rm O_2}\xspace}
\newcommand{\Ntwo}{\ensuremath{\rm N_2}\xspace}
\newcommand{\CHfour}{\ensuremath{\rm CH_4}\xspace}
\newcommand{\Argon}{argon\xspace}
\newcommand{\Ar}{\ensuremath{\rm Ar}\xspace}
\newcommand{\ArR}{\ensuremath{\rm ^{39}Ar}\xspace}
\newcommand{\Kr}{\ensuremath{\rm Kr}\xspace}
\newcommand{\KrR}{\ensuremath{\rm ^{85}Kr}\xspace}
\newcommand{\meth}{methane\xspace}
\newcommand{\He}{\ensuremath{\rm He}\xspace}
\newcommand{\water}{\ensuremath{\rm H_2O}\xspace}
\newcommand{\Htwo}{\ensuremath{\rm H_2}\xspace}
\newcommand{\AMU}{\ensuremath{\rm u}\xspace}
 
\newcommand{\ppm}{\ensuremath{\rm \cdot10^{-6}~g/g}\xspace}
\newcommand{\oneppm}{\ensuremath{\rm 10^{-6}~g/g}\xspace}
\newcommand{\ppb}{\ensuremath{\rm \cdot10^{-9}~g/g}\xspace}
\newcommand{\oneppb}{\ensuremath{\rm 10^{-9}~g/g}\xspace}
\newcommand{\ppt}{\ensuremath{\rm \cdot10^{-12}~g/g}\xspace}
\newcommand{\oneppt}{\ensuremath{\rm 10^{-12}~g/g}\xspace}
\newcommand{\nbb}{\ensuremath{\beta\beta 0\nu}\xspace}
\newcommand{\tbb}{\ensuremath{\beta\beta 2\nu}\xspace}
\newcommand{\xe}{\ensuremath{\rm xenon}\xspace}
\newcommand{\Xe}{\ensuremath{\rm xenon}\xspace}
\newcommand{\K}[1]{\ensuremath{\rm ^{#1}K}\xspace}
\newcommand{\U}[1]{\ensuremath{\rm ^{#1}U}\xspace}
\newcommand{\Th}[1]{\ensuremath{\rm ^{#1}Th}\xspace}
\newcommand{\Pb}[1]{\ensuremath{\rm ^{#1}Pb}\xspace}
\newcommand{\Po}[1]{\ensuremath{\rm ^{#1}Po}\xspace}
\newcommand{\Tl}[1]{\ensuremath{\rm ^{#1}Tl}\xspace}
\newcommand{\Ac}[1]{\ensuremath{\rm ^{#1}Ac}\xspace}
\newcommand{\Bi}[1]{\ensuremath{\rm ^{#1}Bi}\xspace}
\newcommand{\Cs}[1]{\ensuremath{\rm ^{#1}Cs}\xspace}
\newcommand{\KTHU}{K, Th, and U\xspace}
\newcommand{\hnot}{\ensuremath{\rm HNO_3}\xspace}
\newcommand{\degree}{\ensuremath{^{\circ}}\xspace}
\newcommand{\degrees}{\degree}
\newcommand{\singlewidth}{6.8in}
\newcommand{\doublewidth}{3 in}
\newcommand{\figwidth}{\columnwidth}
\newcommand{\ife}[3]{\ifthenelse{\equal{#1}{#2}}{#3}{{}}}


\hypersetup{
    bookmarks=true,         % show bookmarks bar?
    unicode=false,          % non-Latin characters in Acrobat�s bookmarks
    pdftoolbar=true,        % show Acrobat�s toolbar?
    pdfmenubar=true,        % show Acrobat�s menu?
    pdffitwindow=false,     % window fit to page when opened
    pdfstartview={FitH},    % fits the width of the page to the window
    pdftitle={My title},    % title
    pdfauthor={Author},     % author
    pdfsubject={Subject},   % subject of the document
    pdfcreator={Creator},   % creator of the document
    pdfproducer={Producer}, % producer of the document
    pdfkeywords={keyword1} {key2} {key3}, % list of keywords
    pdfnewwindow=true,      % links in new window
    colorlinks=true,       % false: boxed links; true: colored links
    linkcolor=blue,          % color of internal links
    citecolor=black,        % color of links to bibliography
    filecolor=blue,      % color of file links
    urlcolor=cyan           % color of external links
}


\begin{document}
\begin{frontmatter}


\title {Thesis Outline }

\author[umd]{A.~Dobi}
\address[umd]{Physics Department, University of Maryland, College Park MD, USA}



%\begin{abstract}
%We describe a xenon purity analysis system we have developed and used for the LUX dark matter experiment based on a mass spectrometry technique. The device is fully automated, simple, compact and is integrated into the LUX circulation system allowing for hourly, in-situ sampling from several ports. The sensitivity of the spectrometer is enhanced by several orders of magnitude by the presence of a liquid nitrogen cold trap, and many impurity species of interest can be detected at the level of one part-per-billion or better. In the case of \Kr, a troublesome internal background for xenon based dark matter experiments, the sensitivity is sub one part-per-trillion. We have used the technique to screen the LUX xenon before, during, and after the first underground science run, and these measurements have proven useful. This is the second application of the cold trap mass spectrometry technique to an operating physics experiment, first employed for EXO-200.
%\end{abstract}

\end{frontmatter}

%\newpage
%\tableofcontents
\newpage


\section{Dark Matter Overview}
\subsection{Evidence}
\subsection{WIMPs}
\subsection{Experiments}

\section{LUX}
\subsection{About LUX... goals}
\subsection{Xenon dual phase TPC}
\subsection{S1/S2 Pulse areas, timing, XY reconstruction, Energy reconstruction, threshold}
\subsection{Run03 WIMP limit}

\section{Sampling}
\subsection{ Introduction }
\subsection{ Purity Requirements }
\subsection{ Methodology }
\subsection{ Purity Figure of Merit and Calibration}
\subsubsection{ RGA Partial Pressure Measurement}
\subsubsection{ Correcting for RGA Gain Drift }
\subsubsection{ Correcting for Flow Rate}
\subsubsection{ Gain and Flow Corrected Figure of Merit}
\subsubsection{ O2 and H2 Figure of merit}
\subsection{ Applications/Results of Gas Purity Analysis (in LUX)}
\subsubsection{ Xenon Inventory Screening} 
\subsubsection{ Solubility of Impurities in LXe} 
\subsubsection{ 85Kr Monitoring, Ar Monitoring}
\subsubsection{ N2,O2 and getter health monitoring}
\subsubsection{ Gas Run results, Outgassing from Det. components}


\section{XYZ Corrections, Kr83 calibrations}
\subsection{Kr83 source, uniformity, drift velocity and electron attachment vs. field ?}
\subsection{timing separation vs. 9.4 and 32.1 keV LY and QY}
\subsection{calculating S1 XYZ and S2 XYZ corrections and applying them}
\subsection{improvement in resolution with XYZ corrections}
\subsection{trend plots, detector stability}


\section{Energy calibrations (MonoE sources) and Recombination}
\subsection{Recombination theory, counting quanta, intrinsic detector resolution, Fano}
\subsection{Doke plot, finding g1g2 with Xe-activation, Cs, Xray, Kr}
\subsection{Recombination, S1\_stat,S2\_stat and sigmaR vs. Energy for mono-E sources.}
\subsection{Fluctuations in number of exitons for Kr83 and Cs, Xe-act?}

\newpage

\section{Tritium calibration. S2/S1 ER band, LY,QY, Recombination, NEST}
\subsection{Tritium Source, injection and removal, compare with natural methane}
\subsection{natural methane injection with kr83 data showing no change in LY, QY, lifetime}
\subsection{Uniformity, using tritium to calculate Run03 fiducial volume, vs Kr83}
\subsection{Run03 threshold using tritium}
\subsection{ER band calibration, ER/NR discrimination with DD data. Bin by bin. Band Gaussianity}



\section{...maybe a new section. Dealing with a continuous energy source (extension of Ch 5)}
\subsection{Extending methodology of Section 5 to a continuous source}
\subsection{How to un-smear the tritium spectrum. Start with NEST}
\subsection{Extract Recombination, Fluctuations knowing g1,g2,S1\_stat, S2\_stat}
\subsection{Fano Factor from high stats tritium}
\subsection{The standard candle, Kr83 32.1 keV at Zero field}
\subsection{Tritium  LY, QY, Re vs Compton scatter measurements}
\subsection{Apply same method to Cs137 data 150-662 keV}

\section{Conclusion, ER band overview}
\subsection{Patch Tritium + Cs + Dahl's data for  ER band, recombination, fluctuations, LY, LQ}
\subsection{180 vs 100 V/cm Tritium ER band and its fluctuation}
\subsection{Discussion on band mean, width and discrimination predictions at low energy}


%\bibliographystyle{thesisbibstyle}
%\bibliography{TritiumBib}



\end{document}
