\section{Experimental Setup}

The setup of our triated methane calibration technique can be separated into three parts: the tritiated methane source bottle,the injection system,  and the zirconium getter.

\subsection{The tritiated methane source}

The tritiated methane source bottle for our calibration technique consists of a 2250 cc stainless steel bottle which is filled with a mixture of tritiated methane and xenon.  The purpose of this xenon is to serve as a carrier gas for the triaited methane.  The total activity in the source bottle is set by mixing a small amount of tritiated methane from a reservoir into the source bottle via volume sharing.

\subsection{The injection system}

The injection system for our tritiated methane calibration technique consists of a series of expansion volumes which are used to fine tune the amount of CH$_3$T that is injected.  Once the CH$_3$T source bottle is opened it flows through a methane gas purifier (SAES MC1-905F) to remove any sources of potential contamination, such as bare tritium.  The CH$_3$T then flows into the expansions volumes set by the user.   Once the expansion volumes have filled, the flow of xenon in the gas system is diverted through the expansion volumes to sweep the CH$_3$T into the detector.  We continue to flow through the expansion volumes for one hour, which is equivalent to flushing out the expansion volumes over 1000 times, since LUX flows at 20 SLPM and the full 384.5 cc of the expansion volumes are filled with 1590 torr of the xenon and CH$_3$T mixture.  A pump out port allows the expansion volumes to be evacuated in preparation for each use of the injection system.  Note that each injection will lower the total activity in the CH$_3$T source bottle via volume sharing, results in a smaller, yet calculable, injection activity with subsequent injections. 

\subsection{The zirconium getter}

The LUX gas system uses a hot zirconium getter (SAES-PS4MT15R1) downstream of the CH$_3$T injection system to remove CH$_3$T from the xenon.  Extensive R\&D was conducted using a smaller zirconium getter (SAES-PF4C3R1) at the University of Maryland to learn about the CH$_3$T removal efficiency of these purifiers.  Details of these studies is discussed in section \ref{sec:RD} 