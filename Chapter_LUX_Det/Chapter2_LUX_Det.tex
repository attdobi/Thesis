
%Chapter 2

\renewcommand{\thechapter}{2}

\chapter{The LUX Detector}

\section{Introduction}

The LUX experiment is located 4850 ft underground at the Sanford Underground Research Facility. After running for 87 live days in 2013 LUX has recently set the most sensitive limit for a spin independent WIMP scattering cross section [ref] and is expected to achieve five times the sensitivity after a 300 day run in 2015. Nobel elements are ideal candidates for WIMP detection, they are easy to purify and are transparent to their own scintillation light. Xenon is especially favorable do to it's  large atomic mass (131.3 amu) and high liquid phase density ($\rm sim$2.9 kg/l) which provides both an excellent target for coherent WIMP scattering while simultaneously providing excellent shielding from external radioactivity. Xenon also has good radio purity with established techniques to remove and monitor troublesome radio isotopes of argon and krypton found in the atmosphere from which the xenon is distilled [refs]. 
WIMPs being electrically neutral would primarily interact with the xenon target nuclei producing nuclear recoils (NR) whereas typical backgrounds in the detector, gammas and betas, interact with the electrons producing electronic recoils (ER). In liquid xenon ER events can be further discriminated from NR events by about a factor of 1000 by taking the charge to light ratio of the interaction.

\section{The Light and Charge Signals}

When energy is deposited in the active region of the xenon TPC it is converted to excitation, ionization and heat. For a given energy deposit, NR events lose a significant portion of energy to heat leaving less energy available for excitation and ionization than an ER event. Xenon excitons de-excite on the order of nano seconds producing 175nm VUV scintillation light, recombining ion electron pairs also emit scintillation light. The two channels for photon production overlap and sum to produce the primary scintillation signal (S1), arriving within 10s of ns as the photons are collected the  PMT arrays. Electrons which escape recombination with their ion pair feel the effect of the drift field and begin to drift upwards towards the gas phase phase (drift times of 0-324 $\rm \mu s$). Once extracted into the gas the electrons accelerate and electro luminess producing a much larger secondary scintillation signal (S2), proportional the the number of electrons extracted. 
%Show prompt fraction plot to describe event selection.


\section{The LUX TPC}

The LUX detector is a two phase xenon time projection chamber TPC,  [ref LUX inst]. . The detector contains two PMT arrays on the top and bottom with 61 PMTs each for a total of 122 PMTs, with quantum efficiencies ranging from 30-40\%. The active region consists of a 49 cm length between the cathode and gate with a 47 cm diameter of the dodecagonal geometry. The drift field between the cathode and gate is 180 V/cm resulting in an electron drift velocity of 1.51 mm/$\rm \mu s$. The liquid level terminates 5-6mm above the gate grid as the liquid xenon spills over into a weir reservoir. The anode grid is placed 1.0 cm above the gate grid and used to setup the electron extraction field of 6 kV/cm where electrons are removed from the liquid and accelerated causing electroluminessence in the gas phase.  LUX contains a gross 350 kg of xenon of which 250 kg are in the active region and 118 kg fiducial. A schematic of the detector in shown in ...

	WIMPs ... ... ...

%Add self shielding plot... 5 orders of magnitude reduction in the center.



Overview of field

